%\documentclass{cumcmthesis}
\documentclass[withoutpreface,bwprint]{cumcmthesis} %去掉封面与编号页,电子版提交的时候使用。


\usepackage[framemethod=TikZ]{mdframed}
\usepackage{url}   % 网页链接
\usepackage{subcaption} % 子标题
\title{基于求职者心理测试的综合心理测试分析与分类模型研究}
\tihao{B}
\baominghao{4321}
\schoolname{西安交通大学}
\membera{李征}
\memberb{宋昊天}
\memberc{王子隆}
\supervisor{陈磊}
\yearinput{2024}
\monthinput{7}
\dayinput{5}

\begin{document}

 \maketitle
 \begin{abstract}
    心理测试作为重要的研究手段在教育、心理疾病诊断、心理咨询和人才选拔中得到了广泛应用。本文旨在通过对某单位148名求职者的心理测试数据进行分析,研究三种常用量表(症状自评量表SCL-90、职业成熟度量表和大五人格量表)在测试者分类中的应用,并探讨各量表测试结果之间的关联性。
    
    首先,依据单独量表的特点对测试者进行分类,并通过相关分析和交叉验证方法研究各量表结果的联系。其次,基于量表因子和测试数据建立综合评价指标体系,对测试者进行综合分类。随后,通过统计分析方法比较A、B两组测试者在各量表上的差异,揭示分组间的显著性差异。最后,评判心理测试得分总和方法的准确性,并提出改进建议。研究结果为心理测试在人才选拔中的应用提供了科学依据,并为提升心理测试的准确性和客观性提供了参考。

\keywords{综合评价指标,统计分析,分类模型\quad  心理健康评估\quad   症状自评量表,职业成熟度量表,大五人格量表\quad  人才选拔}
\end{abstract}


%\tableofcontents
%\newpage

\section{公式}
\section{参考文献与引用}

参考文献对于一篇正式的论文来说是必不可少的,在建模中重要的参考文献当然应该列出。

%参考文献
\begin{thebibliography}{9}%宽度9
    \bibitem[1]{latex} 
    \bibitem[2]{mathematical-modeling}    全国大学生数学建模竞赛论文格式规范 (2023 年 修改).
    \bibitem[3]{3} \url{https://www.wikipedia.org}
\end{thebibliography}

\newpage
%附录
\begin{appendices}

\section{相关系数矩阵计算}

\begin{lstlisting}[language=python]
    import numpy as np
    import pandas as pd
    import time
    
    start = time.time()
    
    # 读取数据
    read_df = pd.read_excel(r"E:\SAI\TEST.xlsx")
    data = read_df.iloc[2:10, 0:86].values
        print("初始化dis矩阵进度:{}/{}".format(i + 1, n))
    # 下三角复制到上三角
    dis = dis + dis.T - np.diag(dis.diagonal())
    print("初始化dis矩阵进度:{}/{}".format(
    plt.show()
    
    end = time.time()
    print("总运行时间:", end - start)
    
\end{lstlisting}

\section{支撑材料内容组成}


\end{appendices}

\end{document} 