%\documentclass{cumcmthesis}
\documentclass[withoutpreface,bwprint]{cumcmthesis} %去掉封面与编号页,电子版提交的时候使用。


\usepackage[framemethod=TikZ]{mdframed}
\usepackage{url}   % 网页链接
\usepackage{subcaption} % 子标题
\title{基于求职者心理测试的综合心理测试分析与分类模型研究}
\tihao{B}
\baominghao{4321}
\schoolname{西安交通大学}
\membera{李征}
\memberb{宋昊天}
\memberc{王子隆}
\supervisor{陈磊}
\yearinput{2024}
\monthinput{7}
\dayinput{5}

\begin{document}

 \maketitle
 \begin{abstract}
    心理测试作为重要的研究手段在教育、心理疾病诊断、心理咨询和人才选拔中得到了广泛应用。
    本文通过对某单位148名求职者的心理测试数据进行分析,研究三种常用量表(症状自评量表、职业成熟度量表和大五人格量表 )
    在测试者分类中的应用,并探讨各量表测试结果之间的关联性,还建立了综合评价体系,综合考虑三种量表,
    旨在对测试者心理状况、个人行为特征和社会适应等能力进行画像。

    \textbf{对于问题1},我们首先提取汇总了148位测试者的391份心理测试报告中的有价值信息,并剔除掉不符要求的报告。得到数据后,我们使用了k-means聚类算法
    

    肘部,kmeans,0.3-0.4找联系(论文),   



    \textbf{对于问题2},
    评价体系:
    极大变量找分类,熵权法赋权重,
    熵权法:熵值越大,不确定性越大,信息量越小,变异指数越小综合评价能力越弱,权重越小。
    
    \textbf{对于问题3},

    
    \textbf{对于问题4},

\keywords{综合评价指标,统计分析,分类模型\quad  心理健康评估\quad   症状自评量表,职业成熟度量表,大五人格量表\quad 多元分析\quad 人才选拔}
\end{abstract}


%\tableofcontents
%\newpage

\section{问题重述}

心理健康问题近来得到越来越多的关注,而心理测试作为评估心理健康的重要手段在各种领域中得到了广泛应用,如教育、心理疾病诊断、心理咨询和人才选拔等。心理测试利用专业的测量工具和方法,帮助了解个体的心理状况、行为特征和社会适应能力。

目前,针对不同人群和应用场景,存在多种心理测试表,其中包括症状自评量表(SCL-90)、职业成熟度量表和大五人格量表。这些量表在人才招聘领域被广泛使用。SCL-90是全球著名的心理健康测试量表,涵盖广泛的精神病症状学内容,帮助评估个体的心理健康状况;职业成熟度量表旨在测量个体的职业成熟度;大五人格量表基于大五人格理论,评估个体的外向性、宜人性、严谨性、神经质和开放性。

由于心理测试具有间接测量的性质,其评价标准不如物理测量那样绝对和普遍,心理测试结果需要通过多个测试表的交叉验证来提高准确性。然而,交叉验证带来分析上的挑战,因为不同量表的测试项目和侧重点不同,可能导致结果的相悖。

某心理测试机构受单位委托,对该单位148名求职者进行心理测试。测试机构先将测试者随机分成A、B两组,然后对每位求职者采用症状自评量表、职业成熟度量表和大五人格量表三个表进行测试。测试数据和相关分析具体见附件。该单位希望通过测试结果分析,了解每位求职者心理健康状况,人际关系,以便对求职者有一个更加全面了解。

请根据附件所提供的测试数据,通过数学建模完成下列问题:

\textbf{问题1} 三种量表从不同角度对测试者进行分析,请分别按表对148名测试者进行分类。并建立模型,研究三种量表得到结果之间有无联系。

\textbf{问题2} 考虑三种量表因子和测试数据,建立综合评价指标体系,对148名测试者进行分类。

\textbf{问题3} 请建立模型,分析A组测试者与B组测试者之间有无差异。(注:问题1,2中不考虑分组)。

\textbf{问题4} 三种量表均使用得分值总和来研究测试者,请建立模型,对这种方法的准确性加以评判。


通过上述研究,希望能够为求职者心理健康状况的全面了解提供科学依据,为人才选拔中的心理测试应用提供参考。


\section{问题分析}

\subsection{问题一分析}

\subsubsection*{对测试者分类}

在解决问题一时,首先需要进行数据处理。数据收集阶段涉及汇总三张心理测试量表(SCL-90、职业成熟度量表和大五人格量表)中的有效信息,以建立完整的测试数据集。随后的数据清洗过程则专注于移除无用和不相关的数据,例如剔除测试时间小于一分钟的样本。

接下来,为了针对每种量表对测试者进行分类,我们选择三种量表各自具有代表性的5、8、9个参数进行聚类分析。而在选择使用k-means聚类分析前,确定聚类的数目$k$是一个很重要的议题,可以通过肘部法解决。


\subsubsection*{三种量表的联系}
关于三个量表之间的联系,可以通过计算相关系数矩阵来实现,相关系数不仅可以描述变量之间的线性关系强度,还可以用作变量相似性的度量。基于相关系数矩阵,可以进一步进行变量聚类分析,以识别和描述不同量表中相关联的参数组合。

\subsubsection*{三种量表得出结果之间的联系}

最后,为了确定三种量表得到的结果是否存在联系,我们将结合前述的量表参数聚类结果,探索是否存在某种模式或者特定算法能够有效地描述量表间的联系和关联性。这一过程旨在提供对求职者心理健康状况综合评估的科学依据,并为心理测试在人才选拔过程中的应用提供理论支持。

\subsection{问题二分析}

\subsection{问题三分析}

\subsection{问题四分析}

\section{假设与约定}
1. 假设我们使用的所有数据是科学有效,能够较好地反映测试者的心理状况、个人行为特征和社会适应等能力。

2. 假设测试时间小于一分钟的报告不具有参考价值,也就是不考虑测试时间小于一分钟的测试者的报告。

3. 约定只研究同时做过三个人格测试的测试者,不考虑只做过其中一项或两项测试的测试者的报告。

\section{符号说明及名词解释}
表格应具有三线表格式,因此常用 booktabs宏包,其标准格式如\cref{tab:001}~所示。
\begin{table}[!htbp]
    \caption{符号说明}\label{tab:001} \centering
    \begin{tabular}{ccccc}
        \toprule[1.5pt]
        符号 & 意义 & 单位 \\
        \midrule[1pt]
        5 & 269.8 & 0.000674 \\
        10 & 421.0 & 0.001035 \\
        20 & 640.2 & 0.001565 \\
        \bottomrule[1.5pt]
    \end{tabular}
\end{table}

\section{模型建立与求解}

\subsection{问题一求解}

\subsubsection{数据汇总及清洗}
\paragraph*{数据选择及汇总}
在附录提供的148人的391份心理测试报告中,不是每一个人都做了三分测试,同时有的测试报告的测试时间小于1分钟。
由于本题大部分问题旨在探讨三种测试的联系之间的议题,而正常的测试时间需要10-30分钟完成\cite{SCL-90}\cite{The Career Development Quarterly}\cite{The Big Five trait taxonomy},
我们做出假设2和3,即假设测试时间小于一分钟的报告不具有参考价值,舍弃测试时间小于一分钟的测试者的报告,
并约定只研究同时做过三个人格测试的测试者,不考虑只做过其中一项或两项测试的测试者的报告。

为了建立完整的测试数据集,需要选择三种测试中有价值的测验指标。
对于\textbf{症状自评量表SCL-90},选择躯体化、强迫症状、人际关系敏感、抑郁、焦虑、敌对、恐怖、偏执、精神病性9大参量较为合适;
对于\textbf{职业成熟度量表},选择信息应用、职业认知、自我认知、个人调适、职业态度、价值观念、职业选择、条件评估这8大参量比较合适;
而对于\textbf{大五人格量表},应当选取N(神经质),E(外向性),O(开放性),A(宜人性),C(严谨性)这五大指标;

经过两轮筛选和处理,我们最终得到了106人的各自的三大测试数据,这样处理的数据很好地保留了测试结果,同时具有很高的研究价值。


\subsubsection{对测试者分类}


\subsubsection{三种量表的联系}

\subsubsection{三种量表得出结果之间的联系}
\subsection{问题二求解}

\subsection{问题三求解}

\subsection{问题四求解}


\section{模型评价与推广}
\section{参考文献与引用}

%参考文献
\begin{thebibliography}{9}%宽度9
    \bibitem[1]{SCL-90} Derogatis, L. R. (1994). SCL-90-R: Symptom Checklist-90-R. Pearson.
    \bibitem[2]{The Career Development Quarterly} Savickas, M. L. (1999). The Career Development Quarterly.
    \bibitem[3]{The Big Five trait taxonomy} John, O. P., Srivastava, S. (1999). The Big Five trait taxonomy: History, measurement, and theoretical perspectives.
    \bibitem[]{mathematical-modeling}    全国大学生数学建模竞赛论文格式规范 (2023 年 修改).
    \bibitem[]{3} \url{https://www.wikipedia.org}
\end{thebibliography}

\newpage
%附录
\begin{appendices}

\section{相关系数矩阵计算}

\begin{lstlisting}[language=python]
    import numpy as np
    import pandas as pd
    import time

\end{lstlisting}

\section{支撑材料内容组成}


\begin{table}[h!]
    \centering
    \begin{tabular}{c|c|m{10cm}}
        \toprule
        \textbf{文件夹} & \textbf{文件名} & \textbf{主要功能/用途} \\
        \midrule
        \multirow{10}{*}{源代码} 
        & p1\_1\_d.m &  \\
        \cline{2-3}
        & p1\_2\_d.m &  \\
        \cline{2-3}
        & p1\_1.cpp &  \\
        \cline{2-3}
        & p1\_2.cpp &  \\
        \cline{2-3}
        & plot1.m &  \\
        \cline{2-3}
        & plot2.m &  \\
        \cline{2-3}
        & sensitivity1.m &  \\
        \cline{2-3}
        & p2\_4.cpp & \\
        \cline{2-3}
        & problem3\_1.lg4 &  \\
        \cline{2-3}
        & p4.cpp &  \\
        \midrule
        \multirow{2}{*}{数据} 
        & sensitivity.xlsx & 该表格是我们的原始数据表格,该表格用于导入 Matlab 进行计算 \\
        \cline{2-3}
        & cov.xlsx & 该表格是三个量表参量之间的相关系数矩阵 \\
        \bottomrule
    \end{tabular}
    \caption{支持材料内容组成}
    \label{tab:supporting_materials}
\end{table}

\end{appendices}

\end{document} 